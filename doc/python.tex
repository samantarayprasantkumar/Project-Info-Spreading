% !TEX TS-program = pdflatex
% !TEX encoding = UTF-8 Unicode

% This is a simple template for a LaTeX document using the "article" class.
% See "book", "report", "letter" for other types of document.

\documentclass[11pt]{article} % use larger type; default would be 10pt

\usepackage[utf8]{inputenc} % set input encoding (not needed with XeLaTeX)

%%% Examples of Article customizations
% These packages are optional, depending whether you want the features they provide.
% See the LaTeX Companion or other references for full information.

%%% PAGE DIMENSIONS
\usepackage{geometry} % to change the page dimensions
\geometry{a4paper} % or letterpaper (US) or a5paper or....
% \geometry{margin=2in} % for example, change the margins to 2 inches all round
% \geometry{landscape} % set up the page for landscape
%   read geometry.pdf for detailed page layout information

\usepackage{graphicx} % support the \includegraphics command and options

% \usepackage[parfill]{parskip} % Activate to begin paragraphs with an empty line rather than an indent

%%% PACKAGES
\usepackage{amssymb,amsmath}
\usepackage{booktabs} % for much better looking tables
\usepackage{array} % for better arrays (eg matrices) in maths
\usepackage{paralist} % very flexible & customisable lists (eg. enumerate/itemize, etc.)
\usepackage{verbatim} % adds environment for commenting out blocks of text & for better verbatim
\usepackage{subfig} % make it possible to include more than one captioned figure/table in a single float
% These packages are all incorporated in the memoir class to one degree or another...

%%% HEADERS & FOOTERS
\usepackage{fancyhdr} % This should be set AFTER setting up the page geometry
\pagestyle{fancy} % options: empty , plain , fancy
\renewcommand{\headrulewidth}{0pt} % customise the layout...
\lhead{}\chead{}\rhead{}
\lfoot{}\cfoot{\thepage}\rfoot{}

%%% SECTION TITLE APPEARANCE
\usepackage{sectsty}
\allsectionsfont{\sffamily\mdseries\upshape} % (See the fntguide.pdf for font help)
% (This matches ConTeXt defaults)

%%% ToC (table of contents) APPEARANCE
\usepackage[nottoc,notlof,notlot]{tocbibind} % Put the bibliography in the ToC
\usepackage[titles,subfigure]{tocloft} % Alter the style of the Table of Contents
\renewcommand{\cftsecfont}{\rmfamily\mdseries\upshape}
\renewcommand{\cftsecpagefont}{\rmfamily\mdseries\upshape} % No bold!


%%% Code 


\usepackage{xcolor}
\usepackage{listings}
\usepackage{bera}% optional; just for the example

\lstset{
language=Python,
basicstyle=\ttfamily,
otherkeywords={self},             
keywordstyle=\ttfamily\color{blue!90!black},
keywords=[2]{True,False},
keywords=[3]{ttk},
keywordstyle={[2]\ttfamily\color{yellow!80!orange}},
keywordstyle={[3]\ttfamily\color{red!80!orange}},
emph={MyClass,__init__},          
emphstyle=\ttfamily\color{red!80!black},    
stringstyle=\color{green!80!black},
showstringspaces=false            
}




%%% END Article customizations

%%% The "real" document content comes below...

\title{Brief Article}
\author{The Author}
%\date{} % Activate to display a given date or no date (if empty),
         % otherwise the current date is printed 


\begin{document}

\section{Implementation}

\subsection{How did we get our data?} 

To get the data for our simulation, we used the so-called Facebook Graph API. (reference) What does Graph API mean? 
A Facebook Grpah API is a programming tool designed to support better access to conventions on the Facebook social media platform.\footnote{http://www.techopedia.com/definition/28984/facebook-graph-api} It allows you to extract information that is published on facebook, which is exactly what we needed. 

Unfortunately, there is no implementation for Matlab, so we programmed the data-fechting in python.

\subsection{Coding}

\subsubsection{Get access}

For the coding in python we used the facebook sdk for python\footnote{https://github.com/pythonforfacebook/facebook-sdk}. 

First, one needs to get access to the social graph. Therefore, you need to enter an access token\footnote{https://developers.facebook.com/docs/facebook-login/access-tokens/}. 

\begin{lstlisting} 
token = raw_input('Enter your Access Token: ')
anonym = input('Do you want to anonymize your Data? Yes (1) or No (0)? ')
graph = facebook.GraphAPI(token)

\end{lstlisting}

\subsubsection{Get graph}

Having enterd a valid token, we can now access all information this token provide us.  For simplicty, we'll only discuss the implementation of the non-anonymized case in details. [For more details on the anonymization, look at subsection anonymization]

\begin{lstlisting} 
profile = graph.get_object("me")
friends = graph.get_connections("me", "friends")
\end{lstlisting}

In our case, we're interested in data about our own profile, as well as all available information we get from our friends.
The object friends is nothing else but an array that contains all id's of our friends, and for each id, there is another array data, that contains all information about this specific friend.  

\begin{lstlisting}
friend_list = [friend['id'] for friend in friends['data']]
\end{lstlisting}

With this simple for-loop, one can now get a list of all his friends. Since we're not primarly interesseted in our friends, but the connection in between them, we have to access at least the mutualfriends. 

\begin{lstlisting}
mutualfriends = graph.get_connections(tfriend, "mutualfriends")
\end{lstlisting}

The object mutualfriends provides us a list with all mutualfriends of ''me'' and friend ''tfriend'' (which is nothing else but one of the id's we have in the saved in "friend\_list"

Repeat this proceedure for all friends, and we'll get exatly what we're looking for - a graph of all connection of our friends in between of them.

\subsubsection{Get activity}

 

\subsection{Anonymization}


\end{document}
