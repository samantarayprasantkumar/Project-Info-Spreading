\subsection{Homogeneous SIR-model}

Urs

\subsection{Agent-based model}

\subsubsection{Initial condition}

At the beginning of each simulation, all agents are ignorant but one, which is a spreader. In the biggest performed simulation, every one of the 384 nodes was the first spreader 10 times. 

\subsubsection{Determine the meetings}

(See \texttt{talkstep.m} for details.)
\newline
\newline
In order to determine who meets who, the program goes through the vector of agents (1:N) randomly . With a probability corresponding to the activity of that agent, he may meet somebody . The person he meets is determined randomly and must be one he knows and one which is not already meeting another agent in this time step. Agents with more friends also have more meetings(If you don't take the activity into account). 

\subsubsection{Status changes in meetings}

After the meetings are determined, the status of each agent has to change according to the model. In order to be able to determine who infected whom, and how many where infected, a list \texttt{infecpath} was created in each round containing the directed edges of infections.

\subsubsection{cumulative infections}

With a given \texttt{infecpath}, with the directed edges of infections, it is obviously easy to obtain the number of direct infections of every node and also relatively easy to get the number of "cumulative infections", which is the whole subtree of infections of each node. The ladder was obtained in a recursive manner in the following way:

\begin{lstlisting} 
   L=length(infectpath(:,1));
   for i=L:-1:1
      
      p1=infectpath(i,1);
      p2=infectpath(i,2);
      cum_infections(p1)=1+cum_infections(p1)+cum_infections(p2);
   end

\end{lstlisting}



