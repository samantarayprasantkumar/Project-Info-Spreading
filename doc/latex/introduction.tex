\subsection{Introduction and fundamental questions}

Social networks like facebook and twitter are growing fast. Most young people have an account in at least one of them. As many articles, pictures and videos on the internet can be shared easily, it is a very interesting question how this information spreads in such networks. First, the internet may have become the fastest and sometimes most important source of "news" and information besides "traditional" media and second, companies have an increasing interest in targeting the right people with their adverts. 
\\
\\
The simplest way to describe the information spreading uses similarities to epidemiology, which means that the flow of information of one person to another is looked at as an "infection". In this work, we implemented a simple model in a homogeneous way with a set of differential equations and their numerical solutions as well as an agent-based, heterogeneous model, using a real facebook network.
\\
The fundamental questions of this work are:

\begin{itemize}
\item Are there relevant differences in the time evolution of the homogeneous and the agent-based model?

\item Are there indiviuals which are more important to the spreading of information (also called influentials)? Can they be recognized in the sense of position and connectivity in the network?
\end{itemize}

\subsection{Motivation}

As the team of authors consists of two chemists and a mathematician, the personal motivations also have a broad range. We were glad to recognize parallels between our simulations with statistical physics , chemical reaction kinetics, and graph theory. 














