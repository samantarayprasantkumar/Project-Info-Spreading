\subsection{Introduction and fundamental questions}

Social Networks like facebook and twitter are growing very fast. Most young people like students do have an account in one ore more of them. Many articles, pictures and videos on the internet have a direct "share" button. It is a very interesting question how information spreads in such networks. First, the internet may have become the fastest and sometimes most important source of "news" and information beside the "traditional" media and second, companies seem to be very interested in "informing" the right people with their commercials. 
\\
\\
The simplest way to describe the information spreading uses similarities to epidemiology, which means that the flow of information of one person to another is looked at as an "infection". In this work, we implemented a simple model in a homogeneous way with a set of differential equations and their numeric solutions as well as an agent-based, heterogeneous model, using a real facebook network.
\\
The fundamental questions of this work are:

\begin{itemize}
\item Are there relevant differences in the time evolution of the homogeneous and the agent-based network?

\item Are there indiviuals which are more important to the spreading of information? Can they be recognized in the sense of position and connectivity in the network?
\end{itemize}














